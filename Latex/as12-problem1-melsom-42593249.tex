\documentclass[11pt]{article} % use larger type; default would be 10pt

\usepackage[utf8]{inputenc} % set input encoding (not needed with XeLaTeX)

% PAGE DIMENSIONS
\usepackage{geometry} % to change the page dimensions
\geometry{a4paper} % or letterpaper (US) or a5paper or....
\geometry{margin=1in} % for example, change the margins to 2 inches all round
% \geometry{landscape} % set up the page for landscape
%   read geometry.pdf for detailed page layout information

\usepackage{graphicx} % support the \includegraphics command and options
\usepackage{mathtools}
\usepackage{amssymb,amsmath}
\usepackage{gensymb}
\usepackage{float}
\usepackage{epstopdf}

%PACKAGES
\usepackage{booktabs} % for much better looking tables
\usepackage{array} % for better arrays (eg matrices) in maths
\usepackage{paralist} % very flexible & customisable lists (eg. enumerate/itemize, etc.)
\usepackage{verbatim} % adds environment for commenting out blocks of text & for better verbatim
\usepackage{subfig} % make it possible to include more than one captioned figure/table in a single float

% HEADERS & FOOTERS
\usepackage{fancyhdr}
\pagestyle{fancy}
\renewcommand{\headrulewidth}{0pt}
\lhead{phys3071-as12-melsom-42593249}\chead{}\rhead{}
\lfoot{}\cfoot{}\rfoot{}

% SECTION TITLE APPEARANCE
\usepackage{sectsty}
\allsectionsfont{\sffamily\mdseries\upshape} % (See the fntguide.pdf for font help)
\setcounter{secnumdepth}{-1} 

% ToC (table of contents) APPEARANCE
\usepackage[nottoc,notlof,notlot]{tocbibind} % Put the bibliography in the ToC
\usepackage[titles,subfigure]{tocloft} % Alter the style of the Table of Contents
\renewcommand{\cftsecfont}{\rmfamily\mdseries\upshape}
\renewcommand{\cftsecpagefont}{\rmfamily\mdseries\upshape} % No bold!

% END Article customizations

\title{PHYS2041 Lab report 2\\ Single Photon Interference.}
\author{Joshua Melsom - 42593249}
\date{}

\begin{document}


\section{Question 1}
\begin{figure}[!h]
\centering
\includegraphics[scale=0.8]{diagram.eps}
\caption{Visualisation of the FTCS scheme.}
\label{fig:appi}
\end{figure}


Using the notation $u^n _j = u(t_n, x_j)$ for $x_j = x_0 + j\Delta x, \, j=0,1,2...,J$ and $t_n = t_0 + n\Delta t, \, n=0,1,2,...N$ to approximate the PDE $\frac{\partial u}{\partial t} = -D\frac{\partial u}{\partial x}$. Starting with the approximations: \\


$\displaystyle{\frac{\partial u}{\partial t}  \bigg|_{j,n} \approx \frac{u^{n+1}_j - u^n _j}{\Delta t}}$\\ [0.5em]

$\displaystyle{\frac{\partial u}{\partial x} \bigg| _{j,n} \approx \frac{u^n_{j+1} - u^n_{j-1}}{2\Delta x}}$\\[0.5em]

we rearrange our PDE, $i\frac{\partial \psi}{\partial t} = -\frac{\partial ^2 \psi}{\partial x^2} + V\psi $ to the form $\frac{\partial \psi}{\partial t} = i\frac{\partial ^2 \psi}{\partial x^2} -iV\psi $\\

We can now construct a discrete version of our PDE using the above listed approximations. So our equation becomes \\

\noindent $\frac{\psi ^{n+1} _j - \psi^n _j}{\Delta t} = i \bigg ( \frac{\psi ^n _{j+1} - 2\psi^n _j + \psi ^n _{j-1}}{(\Delta x)^2} \bigg ) - iV\psi^n _j$\\

By rearranging the equation so that all $\psi ^{n+1}$ are on one side of the equation and all $\psi ^n$ are on the other side, and expanding the brackets, this becomes. \\

\noindent $\psi ^{n+1} _j= \frac{i \Delta t}{(\Delta x)^2} \psi ^n _{j+1} - \frac{2i \Delta t}{(\Delta x)^2}\psi ^n _j + \frac{i \Delta t}{(\Delta x)^2} \psi ^n _{j-1} - i \Delta t V \psi ^n _j + \psi ^n _j$\\

and now bringing out common factors\\

\noindent $\psi ^{n+1} _j = \frac{i \Delta t}{(\Delta x)^2} \big ( \psi ^n _{j-1} + \psi ^n _{j+1} \big ) + \big ( 1 - \frac{2i \Delta t}{(\Delta x)^2} - i \Delta t V \big ) \psi ^N _j$\\

Now to use the Crank-Nicolson Scheme we add this to \\

\noindent $\psi ^{n+1} _j =\psi ^n _j +  \frac{i \Delta t}{(\Delta x)^2} \big ( \psi ^{n+1} _{j-1} + \psi ^{n+1} _{j} + \psi ^{n+1} _{j+1} \big ) - i \Delta t V \psi ^{n+1} _j $ \\

After adding, and once again grouping the $\psi ^{n+1}$ terms on one side and the $\psi ^n$ terms to the other, we get: \\

\noindent $2\psi ^{n+1} _ j + i\Delta t \psi ^{n+1} _j -  \frac{i \Delta t}{(\Delta x)^2} \big ( \psi^{n+1} _{j+1} - 2\psi ^{n+1} _j + \psi ^{n+1} _{j-1} \big ) =  \frac{i \Delta t}{(\Delta x)^2} \big ( \psi ^n _{j-1} + \psi ^n _{n+1} \big ) + \big ( 2 - i \Delta t V -  \frac{2i \Delta t}{(\Delta x)^2} \big ) \psi ^n _j$\\

Which simplifies to: \\

\noindent $ \big ( 2 + i\Delta t V(x_j) +  \frac{2i \Delta t}{(\Delta x)^2} \big ) \psi ^{n+1} _ j -  \frac{i \Delta t}{(\Delta x)^2} \big ( \psi ^{n+1} _{j-1} + \psi ^{n+1} _{j+1} \big ) =\\ [0.5em] \big ( 2 - i\Delta t V(x_j) -  \frac{2i \Delta t}{(\Delta x)^2} \big ) \psi ^n _ j +  \frac{i \Delta t}{(\Delta x)^2} \big ( \psi ^n _{j-1} + \psi ^n _{j+1} \big )$ \\

Where I have also use the discrete positions to the external potential, $V(x)$.\\

This can be converted to a system of linear equations: \\
\begin{equation}
\noindent
 \begin{bmatrix}
  (2 + i \Delta t V(x_j) + \frac{2i \Delta t}{(\Delta x)^2})&-\frac{i \Delta t}{(\Delta x)^2} &0& \cdots & 0 \\ [0.5em]
  -\frac{i \Delta t}{(\Delta x)^2} &  (2 + i \Delta t V(x_j) + \frac{2i \Delta t}{(\Delta x)^2}) &-\frac{i \Delta t}{(\Delta x)^2}& \cdots &0 \\[0.5em]
0                                                  & -\frac{i \Delta t}{(\Delta x)^2}                                        & \ddots                                       &\vdots & \vdots \\[0.5em]
  \vdots  & \vdots &\cdots & \ddots &  -\frac{i \Delta t}{(\Delta x)^2}  \\[0.5em]
  0 & 0 & -\frac{i \Delta t}{(\Delta x)^2} &(2 + i \Delta t V(x_j) + \frac{2i \Delta t}{(\Delta x)^2})
 \end{bmatrix}
\nonumber
\end{equation}\\

\begin{equation}
 \begin{bmatrix}
       \psi ^{n+1} _1  \\[0.5em]
       \psi ^{n+1} _2 \\[0.5em]
       \vdots \\[0.5em]
       \psi ^{n+1} _{J-2} \\[0.5em]
        \psi ^{n+1} _{J-1} 
     \end{bmatrix} = 
 \begin{bmatrix}
       (2 - i \Delta t V(x_1) - \frac{2i \Delta t}{(\Delta x)^2}) \psi ^n _1 + \frac{i \Delta t}{(\Delta t)^2} (\psi ^n _0 + \psi ^n _2 ) + \frac{i \Delta t}{(\Delta x)^2} \psi^{n+1} _0 \\[0.5em]
       (2 - i \Delta t V(x_2) - \frac{2i \Delta t}{(\Delta x)^2}) \psi ^n _2 + \frac{i \Delta t}{(\Delta t)^2} (\psi ^n _1 + \psi ^n _3 )                       \\[0.5em]
       \vdots \\[0.5em]
       (2 - i \Delta t V(x_{J-2}) - \frac{2i \Delta t}{(\Delta x)^2}) \psi ^n _{J-2} + \frac{i \Delta t}{(\Delta t)^2} (\psi ^n _{J-3} + \psi ^n _{J-4} )\\[0.5em]
       (2 - i \Delta t V(x_{J-1}) - \frac{2i \Delta t}{(\Delta x)^2}) \psi ^n _{J-1} + \frac{i \Delta t}{(\Delta t)^2} (\psi ^n _{J-2} + \psi ^n _J ) + \frac{i \Delta t}{(\Delta x)^2} \psi^{n+1} _J \\[0.5em]
     \end{bmatrix}
\nonumber
\end{equation} \\

There the final part of the first and last row are zero so cancel off leaving the RHS $\psi ^{n+1} _j$ completely dependant on the LHS $\psi^n _j $ for eveny time step.\\

Because this is of the form $\mathbf{A} \underline{x} =\underline{b}$ we can solve the equivalent statement $ \underline{x} =\mathbf{A}^{-1} \underline{b}$. \\

We use the function \emph{complex\_tridiag.c} to solve this.

\end{document}